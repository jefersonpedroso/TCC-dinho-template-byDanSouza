\documentclass[
	% -- opções da classe memoir --
	12pt,				% tamanho da fonte
	openright,			% capítulos começam em pág ímpar (insere página vazia caso preciso)
	oneside,			% para impressão em verso e anverso. Oposto a oneside
	a4paper,			% tamanho do papel. 
	% -- opções da classe abntex2 --
	%chapter=TITLE,		% títulos de capítulos convertidos em letras maiúsculas
	%section=TITLE,		% títulos de seções convertidos em letras maiúsculas
	%subsection=TITLE,	% títulos de subseções convertidos em letras maiúsculas
	%subsubsection=TITLE,% títulos de subsubseções convertidos em letras maiúsculas
	% -- opções do pacote babel --
	english,			% idioma adicional para hifenização
	french,             % idioma adicional para hifenização
	spanish,			% idioma adicional para hifenização
	brazil,				% o último idioma é o principal do documento
	]{abntex2}


% ---
% PACOTES
% ---

\usepackage{lmodern}			% Usa a fonte Latin Modern
\usepackage[T1]{fontenc}		% Selecao de codigos de fonte.
\usepackage[utf8]{inputenc}		% Codificacao do documento (conversão automática dos acentos)
\usepackage{indentfirst}			% Indenta o primeiro parágrafo de cada seção.
\usepackage{color}				% Controle das cores
\usepackage{graphicx}			% Inclusão de gráficos
\usepackage{subfig}
\usepackage{listings}
\usepackage{float}
\usepackage[final]{pdfpages}

\definecolor{mygreen}{RGB}{28,172,0} % color values Red, Green, Blue
\definecolor{mylilas}{RGB}{170,55,241}

\usepackage{microtype} 			% para melhorias de justificação

% Pacotes adicionais, usados no anexo do modelo de folha de identificação

\usepackage{multicol}
\usepackage{multirow}

% Pacotes adicionais, usados apenas no âmbito do Modelo Canônico do abnteX2

\usepackage{lipsum}				% para geração de dummy text

% Pacotes de citações

\usepackage[brazilian,hyperpageref]{backref}	% Paginas com as citações na bibl
\usepackage[alf]{abntex2cite}				% Citações padrão ABNT

% --- 
% CONFIGURAÇÕES DE PACOTES
% --- 

% ---
% Configurações do pacote backref
% Usado sem a opção hyperpageref de backref
\renewcommand{\backrefpagesname}{Citado na(s) página(s):~}
% Texto padrão antes do número das páginas
\renewcommand{\backref}{}
% Define os textos da citação
\renewcommand*{\backrefalt}[4]{
	\ifcase #1 %
		Nenhuma citação no texto.%
	\or
		Citado na página #2.%
	\else
		Citado #1 vezes nas páginas #2.%
	\fi}%
% ---

% ---
% Informações de dados para CAPA e FOLHA DE ROSTO
% ---
\titulo{Camada Física\\Modulações e Codificação de Linha}
\autor{Daniel Henrique Camargo de Souza\\Augusto Daniel Rodrigues}
\local{Florianópolis, SC - Brasil}
\data{15 de Agosto de 2018}
\instituicao{%
  Instituto Federal de Santa Catarina - IFSC
  \par
  Departamento Acadêmico de Eletrônica - DAELN
  \par
}
\tipotrabalho{Relatório técnico}
% O preambulo deve conter o tipo do trabalho, o objetivo, 
% o nome da instituição e a área de concentração 
\preambulo{Relatorio Técnico referênte ao estudo da Camada Física, Modulações e Codificação de Linha, da disciplina de Redes de Computadores - RCP22108}
% ---

% ---
% Configurações de aparência do PDF final

% alterando o aspecto da cor azul
\definecolor{blue}{RGB}{41,5,195}

% informações do PDF
\makeatletter
\hypersetup{
     	%pagebackref=true,
	pdftitle={\@title}, 
	pdfauthor={\@author},
    	pdfsubject={\imprimirpreambulo},
	%pdfcreator={LaTeX with abnTeX2},
	%pdfkeywords={abnt}{latex}{abntex}{abntex2}{relatório técnico}, 
	colorlinks=true,       		% false: boxed links; true: colored links
    	linkcolor=blue,		          	% color of internal links
    	citecolor=blue,        		% color of links to bibliography
    	filecolor=magenta,      		% color of file links
	urlcolor=blue,
	bookmarksdepth=4
}
\makeatother
% --- 

% --- 
% Espaçamentos entre linhas e parágrafos 
% --- 

% O tamanho do parágrafo é dado por:
\setlength{\parindent}{1.3cm}

% Controle do espaçamento entre um parágrafo e outro:
\setlength{\parskip}{0.2cm}  % tente também \onelineskip

% ---
% compila o indice
% ---
\makeindex
% ---

% ---------------------------------------------------------------------
% Início do documento
% ---------------------------------------------------------------------
\begin{document}

\lstset{language=Matlab,%
    %basicstyle=\color{red}, 		%normal fontsize
    %basicstyle=\ttfamily\footnotesize 	%fontsize small
    basicstyle=\ttfamily\scriptsize,		%fontsize verysmall
    %breaklines=false,%
    morekeywords={matlab2tikz},
    keywordstyle=\color{blue},%
    morekeywords=[2]{1}, keywordstyle=[2]{\color{black}},
    identifierstyle=\color{black},%
    stringstyle=\color{mylilas},
    commentstyle=\color{mygreen},%
    showstringspaces=false,%without this there will be a symbol in the places where there is a space
    numbers=left,%
    numberstyle={\tiny \color{black}},% size of the numbers
    numbersep=9pt, % this defines how far the numbers are from the text
    emph=[1]{for,end,break},emphstyle=[1]\color{red}, %some words to emphasise
    %emph=[2]{word1,word2}, emphstyle=[2]{style},    
}


% Retira espaço extra obsoleto entre as frases.
\frenchspacing 

% ----------------------------------------------------------
% ELEMENTOS PRÉ-TEXTUAIS
% ----------------------------------------------------------
% \pretextual

% ---
% Capa
% ---
\imprimircapa
% ---

% ---
% Folha de rosto
% (o * indica que haverá a ficha bibliográfica)
% ---
% \imprimirfolhaderosto*
\imprimirfolhaderosto
% ---

% ---
% Anverso da folha de rosto:
% ---

% resumo na língua vernácula (obrigatório)
% \setlength{\absparsep}{18pt} % ajusta o espaçamento dos parágrafos do resumo

% \begin{resumo}
 
% As fontes de alimentação são equipamentos essenciais para o funcionamento de circuitos e sistemas eletrônicos. O propósito deste trabalho é a implementação de % uma fonte de alimentação linear para as tensões fixas de 12VDC, -12VDC, 5VDC e uma saída ajustável de 0 até 20VDC, desta forma, este equipamento poderá ser % utilizado em uma grande faixa de circuitos e projetos eletrônicos. As etapas de implementação são: O projeto, como cálculos básicos e a simulação; A montagem % % com as considerações e cuidados a serem tomados; e as medidas, com os resultados e as comparações de desempenho. 

%  \noindent
%  \textbf{Palavras-chaves}: Fonte linear. 

% \end{resumo}
% ---

% ---
% inserir lista de ilustrações
% ---
% \pdfbookmark[0]{\listfigurename}{lof}
% \listoffigures*
% \cleardoublepage
% ---

% ---
% inserir lista de tabelas
% ---
% \pdfbookmark[0]{\listtablename}{lot}
% \listoftables*
% \cleardoublepage
% ---

% ---
% inserir lista de abreviaturas e siglas
% ---
% \begin{siglas}
%   \item[VDC] \textit{Voltage Direct Current} - Tensão Contínua
%   \item[VAC] \textit{Voltage Alternating Current} - Tensão Alternada
%   \item[CI] Circuito Integrado
%   \item[CC] Corrente Contínua
%   \item[CA] Corrente Alternada
%   \item[IFSC] Instituto Federal de Educação Ciência e Tecnologia de Santa Catarina
%   \item[PCI] Placa de Circuito Impresso
%   \item[ABNT] Associação Brasileira de Normas Técnicas
%   \item[abnTex] Normas para \LaTeX
%   \end{siglas}
% ---

% ---
% inserir lista de símbolos
% ---
% \begin{simbolos}
%   \item[$cm$] Centímetros - Unidade de comprimento
%   \item[$cm^{2}$] Centímetros Quadrados - Unidade de área
%   \item[$VA$] Volt-Ampere - Unidade de potência elétrica
%   \item[$W$] Watt's - Unidade de potência elétrica
%   \item[$V$] Volts - Unidade de potencial elétrico
% 	\item[$A$] Ampere - Unidade de Corrente Elétrica
% 	\item[$\Omega$] Ohms - Unidade de resistência elétrica
% 	\item[$H$] Henry - Unidade de indutância elétrica
% 	\item[$\,^{\circ}\mathrm{C}$] Grau Celcius - Unidade de temperatura
% \end{simbolos}
% ---

% ---
% inserir o sumario
% ---
\pdfbookmark[0]{\contentsname}{toc}
\tableofcontents
\cleardoublepage
% ---

% ----------------------------------------------------------
% ELEMENTOS TEXTUAIS
% ----------------------------------------------------------
\textual

%\part{Introdução}
% \include{./text/intro}
 \include{./text/estudo_implementa}
% \include{./text/resultado}
% ----------------------------------------------------------
% ELEMENTOS PÓS-TEXTUAIS
% ----------------------------------------------------------
% \postextual

% ----------------------------------------------------------
% Referências bibliográficas
% ----------------------------------------------------------
% \bibliography{bib_eletronic.eng}

% ----------------------------------------------------------
% Glossário
% ----------------------------------------------------------
%
% Consulte o manual da classe abntex2 para orientações sobre o glossário.
%
%\glossary

% ----------------------------------------------------------
% Apêndices
% ----------------------------------------------------------
 \begin{apendicesenv}
 \chapter{Códigos Elaborados}
\section{Função para o cálculo da Taxa de Erro - bitError.m}
\begin{lstlisting}
% Function to calculate error tax in two text arquives
% Parameters:   bitErrorTax = bitError();
%               no input parameters

function bitErrorTax = bitError()
    %% Open the files
    % Read input file
    file = fopen('entrada.txt', 'r');
    input=fread(file);
    fclose(file);
    % Read output file
    file   = fopen('saida.txt', 'r');
    output=fread(file);
    fclose(file);

    %% Sweeping to find error
    errorBit = 0;
    for i=1:length(input)
        if input(i) ~= output(i)
            errorBit = errorBit + 1;              
        end
    end

    %% Finding the Tax
    bitErrorTax = errorBit/length(input);
end
\end{lstlisting}

\section{Cálculo das Taxas de Erro - bitErrorTax\_ BFSK\_ 4FSK\_ BPSK.m}
\begin{lstlisting}
% SNRs = [1 5 20]
% BFSK, 4-FSK, BPSK

SNRs = [0.5 1 5 10 15 20];
bitErrorTax_bfsk = zeros(0,length(SNRs));
bitErrorTax_4fsk = zeros(0,length(SNRs));
bitErrorTax_bpsk = zeros(0,length(SNRs));
for i=1:length(SNRs)
    trans_bfsk(SNRs(i));
    bitErrorTax_bfsk(i) = bitError();
end

for i=1:length(SNRs)
    trans_4fsk(SNRs(i));
    bitErrorTax_4fsk(i) = bitError();
end

for i=1:length(SNRs)    
    trans_bpsk(SNRs(i));
    bitErrorTax_bpsk(i) = bitError();
end

FID = fopen('bitErrorTax.txt','w');
fprintf(FID, 'bitErrorTax_bfsk = [');
for i=1:length(bitErrorTax_bfsk)
    fprintf(FID, '%d ',bitErrorTax_bfsk(i));
end
fprintf(FID, ']\n');

fprintf(FID, 'bitErrorTax_4fsk = [');
for i=1:length(bitErrorTax_4fsk)
    fprintf(FID, '%d ',bitErrorTax_4fsk(i));
end
fprintf(FID, ']\n');

fprintf(FID, 'bitErrorTax_bpsk = [');
for i=1:length(bitErrorTax_bpsk)
    fprintf(FID, '%d ',bitErrorTax_bpsk(i));
end
fprintf(FID, ']');
ST1 = fclose(FID);
\end{lstlisting}

\section{Codificação de linha - URZ}
\begin{lstlisting}
function [t,x,dt] = urz(bits, bitrate)
% URZ Encode bit string using unipolar RZ code.
%   [T, X] = URZ(BITS, BITRATE) encodes BITS array using unipolar RZ
%   code with given BITRATE. Outputs are time T and encoded signal
%   values X.

% Copyright (c) 2013 Yuriy Skalko <yuriy.skalko@gmail.com>

T = length(bits)/bitrate; % full time of bit sequence
n = 200;
N = n*length(bits);
dt = T/N;
t = 0:dt:T;
x = zeros(1,length(t)); % output signal

for i = 0:length(bits)-1
  if bits(i+1) == 1
    x(i*n+1:(i+0.5)*n) = 1;
    x((i+0.5)*n+1:(i+1)*n) = 0;
  else
    x(i*n+1:(i+1)*n) = 0;
  end
end
\end{lstlisting}

\section{Codificação de linha - UNRZ}
\begin{lstlisting}
function [t,x,dt] = unrz(bits, bitrate)
% UNRZ Encode bit string using unipolar NRZ code.
%   [T, X] = UNRZ(BITS, BITRATE) encodes BITS array using unipolar NRZ
%   code with given BITRATE. Outputs are time T and encoded signal
%   values X.

% Copyright (c) 2013 Yuriy Skalko <yuriy.skalko@gmail.com>

T = length(bits)/bitrate; % full time of bit sequence
n = 200;
N = n*length(bits);
dt = T/N;
t = 0:dt:T;
x = zeros(1,length(t)); % output signal

for i = 0:length(bits)-1
  if bits(i+1) == 1
    x(i*n+1:(i+1)*n) = 1;
  else
    x(i*n+1:(i+1)*n) = 0;
  end
end
\end{lstlisting}

\section{Codificação de linha - PRZ}
\begin{lstlisting}
function [t,x,dt] = prz(bits, bitrate)
% PRZ Encode bit string using polar RZ code.
%   [T, X] = PRZ(BITS, BITRATE) encodes BITS array using polar RZ
%   code with given BITRATE. Outputs are time T and encoded signal
%   values X.

% Copyright (c) 2013 Yuriy Skalko <yuriy.skalko@gmail.com>

T = length(bits)/bitrate; % full time of bit sequence
n = 200;
N = n*length(bits);
dt = T/N;
t = 0:dt:T;
x = zeros(1,length(t)); % output signal

for i = 0:length(bits)-1
  if bits(i+1) == 1
    x(i*n+1:(i+0.5)*n) = 1;
    x((i+0.5)*n+1:(i+1)*n) = 0;
  else
    x(i*n+1:(i+0.5)*n) = -1;
    x((i+0.5)*n+1:(i+1)*n) = 0;
  end
end
\end{lstlisting}

\section{Codificação de linha - Manchester}
\begin{lstlisting}
function [t,x,dt] = manchester(bits, bitrate)
% MANCHESTER Encode bit string using Manchester code.
%   [T, X] = MANCHESTER(BITS, BITRATE) encodes BITS array using Manchester
%   code with given BITRATE. Outputs are time T and encoded signal
%   values X.

% Copyright (c) 2013 Yuriy Skalko <yuriy.skalko@gmail.com>

T = length(bits)/bitrate; % full time of bit sequence
n = 200;
N = n*length(bits);
dt = T/N;
t = 0:dt:T;
x = zeros(1,length(t)); % output signal

for i = 0:length(bits)-1
  if bits(i+1) == 1
    x(i*n+1:(i+0.5)*n) = 1;
    x((i+0.5)*n+1:(i+1)*n) = -1;
  else
    x(i*n+1:(i+0.5)*n) = -1;
    x((i+0.5)*n+1:(i+1)*n) = 1;
  end
end
\end{lstlisting}

\section{Codificação de linha - Gráficos}
\begin{lstlisting}
% Demo of using different line codings
bits = [1 0 1 1 1 1 1 1 0 1 0 0 1 1 0 1 0 1 0 0];
bitrate = 1; % bits per second

%%UNRZ --------------------------------------
figure(1);
[t,s,dt] = unrz(bits,bitrate);
plot(t,s,'LineWidth',3);
axis([0 t(end) -0.1 1.1])
grid on;
title(['Unipolar NRZ: [' num2str(bits) ']']);

%Averange Value
avgVal=0;
for i=1:length(s)
    avgVal = avgVal + s(i);
end
averange_UNRZ = avgVal/length(s);

figure(2);
Fs = 1/dt;
pwelch(s,[],[],[],Fs);
title('Densidade Espectral UNRZ');

%%URZ ---------------------------------------
figure(3);
[t,s,dt] = urz(bits,bitrate);
plot(t,s,'LineWidth',3);
axis([0 t(end) -0.1 1.1])
grid on;
title(['Unipolar RZ: [' num2str(bits) ']']);

%Averange Value
avgVal=0;
for i=1:length(s)
    avgVal = avgVal + s(i);
end
averange_URZ = avgVal/length(s);

figure(4);
Fs = 1/dt;
pwelch(s,[],[],[],Fs);
title('Densidade Espectral URZ');

%%PRZ ---------------------------------------
figure(5);
[t,s,dt] = prz(bits,bitrate);
plot(t,s,'LineWidth',3);
axis([0 t(end) -1.1 1.1])
grid on;
title(['Polar RZ: [' num2str(bits) ']']);

%Averange Value
avgVal=0;
for i=1:length(s)
    avgVal = avgVal + s(i);
end
averange_PRZ = avgVal/length(s);

figure(6);
Fs = 1/dt;
pwelch(s,[],[],[],Fs);
title('Densidade Espectral PRZ');

%%Manchester --------------------------------
figure(7);
[t,s] = manchester(bits,bitrate);
plot(t,s,'LineWidth',3);
axis([0 t(end) -1.1 1.1])
grid on;
title(['Manchester: [' num2str(bits) ']']);

%Averange Value
avgVal=0;
for i=1:length(s)
    avgVal = avgVal + s(i);
end
averange_Manchester = avgVal/length(s);

figure(8);
Fs = 1/dt;
pwelch(s,[],[],[],Fs);
title('Densidade Espectral Manchester');
\end{lstlisting}

 \end{apendicesenv}


% ----------------------------------------------------------
% Anexos
% ----------------------------------------------------------
%\begin{anexosenv}
%\chapter{Datasheets dos Fabricantes}
\section{Controlador PWM NCP3020A - \textit{\textbf{ON Semiconductor}}}
%\includepdf[pages=-]{./pdf/NCP3020}
\section{Controlador LTC3854 - \textit{\textbf{Linear Technology}}}
%\includepdf[pages=-]{./pdf/LTC3854}
\section{Controlador Buck MIC2103 - \textit{\textbf{Microchip}}}
%\includepdf[pages=-]{./pdf/MIC210X}

%\end{anexosenv}

%-----------------------------------------------------------
% INDICE REMISSIVO
%-----------------------------------------------------------

% \phantompart

% \printindex

\end{document}
